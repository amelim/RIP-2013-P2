\documentclass[12pt]{article}
\usepackage{amsmath}
\usepackage{amssymb,amsmath}
\usepackage{graphics,graphicx,amssymb,verbatim,color}
\usepackage{epstopdf,amsxtra,epsfig, psfrag, subfig} %subfigure
\usepackage{titlesec}
\usepackage[toc,page]{appendix}
\DeclareGraphicsRule{.tif}{png}{.png}{`convert #1 `basename #1 .tif`.png}

%%% Set the margins %%%
\setlength{\topmargin}{-.25in}
\setlength{\headheight}{0in}
\setlength{\headsep}{.5in}
\setlength{\textheight}{8.75in}
\setlength{\textwidth}{7in}
\setlength{\oddsidemargin}{-.125in}
\setlength{\evensidemargin}{0in}

%%% Set paragraph style %%%
\setlength{\parindent}{0pt} 
\setlength{\parskip}{2ex}

%%% Helper Macros %%%
\newcommand{\note}{\textcolor{red}}
\providecommand{\todo}[0]{\textcolor{blue}{TO DO}}
\providecommand{\T}[0]{\textsf{T}}
\providecommand{\ul}[1]{\underline{#1}}
\providecommand{\abs}[1]{\lvert#1\rvert}
\providecommand{\norm}[1]{\left\lVert#1\right\rVert}
\providecommand{\vector}[2]{[#1_1,\ldots,#1_{#2}]} % Vector
\providecommand{\argmin}{\operatornamewithlimits{argmin}} % Minimum argument
\providecommand{\argmax}{\operatornamewithlimits{argmax}} % Maximum argument
\providecommand{\N}{\mathbb{N}} % Natural Numbers
\providecommand{\Z}{\mathbb{Z}} % Integers Numbers
\providecommand{\R}{\mathbb{R}} % Real Numbers
\providecommand{\Im}{\mathcal{R}} % Range Space
\providecommand{\Ker}{\mathcal{N}} % Null Space

\newcommand{\eqnbox}[1]
{
\begin{equation*}
\addtolength{\fboxsep}{1pt}
\boxed{
\begin{split}
#1
\end {split}
}
\end{equation*}
}

%%% Homework Document Macros %%%
% Homework specifics
\newcommand{\hwName}{Assignment 2}
\newcommand{\dueDate}{11/25/2013}
\newcommand{\myFirstName}{Daniel Pickem, Andrew Melim, Jarius Tillman, Matheus Svolenski}
\newcommand{\myLastName}{}
\newcommand{\myIDNumber}{}
\newcommand{\courseNum}{CS 4649/7649}
\newcommand{\courseName}{RIP - Robot Intelligence - Planning}
\newcommand{\courseTerm}{Fall 2013}

% Full Name
\newcommand{\myFullName}{\myFirstName~\myLastName~}

% Document Header - 1st page only
\newcommand{\docHeader}
{
{\myFullName} \\
\noindent\rule[1pt]{\textwidth}{1pt}
\noindent{\courseName ~ \courseNum ~ \courseTerm} \hfill 
\noindent{\hwName} \hfill {\dueDate}
\noindent\rule[8pt]{\textwidth}{1pt}
}

% Running Header - every page
\newcommand{\runningHeader}{\markright{\it{\courseNum ~ \hwName \hfill \myLastName~~~}}}

% New problem
\titleformat{\section}{\large\bfseries}{\thesection}{1em}{}
\newcommand{\problem}[2]{\section*{Problem {#1}~-~{#2}}}
\newcommand{\problemPart}[1]{\subsection*{(#1)}}

% ------------------------------------------------------------------------------------------------------------
% BEGIN DOCUMENT
% ------------------------------------------------------------------------------------------------------------
\begin{document}
\runningHeader
\pagestyle{myheadings}
\thispagestyle{empty}
\docHeader

Deliverables include the following: 
\begin{itemize}
 \item Email the Instructor AND TA a PDF summary that describes your work, results and answers to all the questions posed below. Thoroughness in analysis, plots, great movies
and answers in the reports are the primary component of your grade!
 \item A zip file that includes your code in an organized form 
 \item A README file describing how to run the code to re-create the results (DO NOT send movies).
 \item A one page summary describing what each group member contributed.
 \item Print the summary and bring it to class on November 26th as well as a USB stick with movies.
\end{itemize}

% ++++++++++++++++++++++++++++++++++++++++++++++++
\problem{1}{Jacobian Control}
\label{sec:problem_1}
% ++++++++++++++++++++++++++++++++++++++++++++++++


% ++++++++++++++++++++++++++++++++++++++++++++++++
\problem{2}{RRTs}
\label{sec:problem_2}
% ++++++++++++++++++++++++++++++++++++++++++++++++

% ++++++++++++++++++++++++++++++++++++++++++++++++
\problem{3}{Task Constraints}
\label{sec:problem_3}
% ++++++++++++++++++++++++++++++++++++++++++++++++

% ++++++++++++++++++++++++++++++++++++++++++++++++
\problem{4}{Questions}
\label{sec:problem_4}
% ++++++++++++++++++++++++++++++++++++++++++++++++

\newpage
\begin{center}
\Huge{Summary} 
\end{center}
This section summarizes the contribution of each team member. 

\begin{enumerate}
  \item Matheus Svolenski
    \begin{itemize}
     \item ???
    \end{itemize}

  \item Jarius Tillman
   \begin{itemize}
     \item ???
    \end{itemize}
  \item Andrew Melim
    \begin{itemize}
     \item ???
    \end{itemize}

  \item Daniel Pickem
    \begin{itemize}
     \item ???
    \end{itemize}
\end{enumerate}

% ##########################################
% 		Bibliography
% ##########################################
\begin{thebibliography}{50}
  %\bibitem[label]{key}.
  \bibitem{Russell2010} Russell, Stuart J. and Norvig, Peter, \textsl{Artificial Intelligence: A Modern Approach}, Pearson Education, 2010
  \bibitem{Blackbox} BlackBox Planner, \textsl{http://www.cs.rochester.edu/~kautz/satplan/blackbox/\#super}
  \bibitem{FastForward} Fast Forward Planner, \textsl{http://fai.cs.uni-saarland.de/hoffmann/ff.html}
\end{thebibliography}

\newpage
\begin{appendix}
???
\end{appendix}

\newpage
\end{document}
